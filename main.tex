\documentclass[reqno]{amsart}

\usepackage{
  amsmath, amsthm, amssymb, mathtools, dsfont, units,          % Math typesetting
  graphicx, wrapfig, subfig, float,                            % Graphics
  listings, color, inconsolata, pythonhighlight,               % Code
  fancyhdr, hyperref, enumerate, enumitem, framed, xcolor}   % Other
  
\usepackage{mathrsfs}

\usepackage{pgfplots}

\usepackage{tikz-cd}
\tikzcdset{scale cd/.style={every label/.append style={scale=#1},
    cells={nodes={scale=#1}}}}

% Anchor footnotes to bottom
\usepackage[bottom]{footmisc}

% Adjust line spacing
\renewcommand{\baselinestretch}{1.07}

% Remove paragraph indentation
\setlength\parindent{0pt}

% Spacing between paragraphs
\setlength{\parskip}{1.5mm}

% Allow multi-line equations to break to next page
\allowdisplaybreaks

% Set page margins
\usepackage{geometry}\geometry{letterpaper,
  left=1.45in, right=1.45in,
  top=1.1in, bottom=1in,
  headsep=.2in }

% -- Frame settings (for problem statement) --
\setlength\FrameSep{0.55em}
\setlength\OuterFrameSep{\partopsep}

% Command for homework problems. Likely not useful for lecture notes, but I included it just in case.
% Usage: '\question{Problem number}{Problem statement}'
\newcommand{\question}[2]{\begin{framed}\noindent \textbf{Question #1 --} #2\end{framed}}

\newcommand{\problem}[2]{\begin{framed}\noindent \textbf{Problem #1 -- }#2\end{framed}}

\newcommand{\assignment}[2]{\begin{framed}\noindent \textbf{Assignment #1 -- }#2\end{framed}}

% -- Flush left for 'enumerate' numbers
%\setlist[enumerate]{wide=0pt, leftmargin=21pt, labelwidth=0pt, align=left}
%\setenumerate[1]{label={(\arabic*)}}

% -- Settings for internal/external links --
\hypersetup{colorlinks=true, linkcolor=magenta}

% -- Make reference section title font smaller --
\renewcommand{\refname}{\normalsize\bf{References}}

% -- Settings for code listings -- You likely won't use this so ignore it
\definecolor{greenText}{rgb}{0.5, 0.7, 0.5}
\definecolor{greyText}{rgb}{0.5, 0.5, 0.5}
\definecolor{codeFrame}{rgb}{0.7, 0.5, 0.5}
\definecolor{backgroundColor}{rgb}{0.95, 0.95, 0.92}
\lstdefinestyle{code} {
% Uncomment the next line (remove %) to add green background to listing
  %backgroundcolor=\color{backgroundColor},
% Comment the next line (add %) and the next to remove green border
  frame=single,
  rulecolor=\color{greenText},
  numberstyle=\tiny\color{greyText},
  commentstyle=\color{greenText},
  basicstyle=\linespread{1}\ttfamily\footnotesize,
  keywordstyle=\ttfamily\footnotesize,
  showstringspaces=false,
  numbers=left,
  numbersep=8pt,
  xleftmargin=1.95em,
  framexleftmargin=1.6em }
\lstset{style=code} 

% -- Math/Statistics/Physics commands -- Most regular commands should still work, but these make typing a lot faster. I suggest learning them over time and adding your own commands/changing these when you get comfortable with the newcommand structure

% Add a reference number to a single line of a multi-line equation in align/gather environment. In place of 'name here' you may use any name you want. Any other instances of 'numberthis' using the same 'name here' will increase in count. Otherwise, a new 'name here' will use a different counter, starting back at (1).
% Usage: '\numberthis\label{(name here)}'
\newcommand\numberthis{\addtocounter{equation}{1}\tag{\theequation}}

% Shortcuts for bold/mathcal math symbols; \bg is for bold Greek letters
\let\bf\mathbf
\let\bg\boldsymbol
\let\mc\mathcal
\newcommand{\ms}{\mathscr}
\let\mf\mathfrak

% Normal distribution: first argument is the mean, second argument is the variance
% Usage: '\normal{\mu}{\sigma}'
\newcommand{\normal}[2]{\mathcal{N}(#1,#2)}

% Uniform distribution: first parameter is left endpoint, second parameter is right endpoint
% Usage: '\unif{0}{1}'
\newcommand{\unif}[2]{\text{Uniform}(#1,#2)}

% Norm: ||#1||_#2; first parameter is vector, second parameter is deliminator for vector space
% Usage: '\norm{f}{p}'
\newcommand{\norm}[2]{\left\|#1\right\|_{#2}}

% Sum: first parameter is bottom of sum, second parameter is top
% Usage: '\s{n=1}{\infty}'
\newcommand{\s}[2]{\sum_{#1}^{#2}}

% Integral: first parameter is bottom, second is top
% Usage: '\i{0}{1}'
\renewcommand{\i}[2]{\int_{#1}^{#2}}

% Integral Variants:
% Usage '\it{0}{1}{f(t)}'
\renewcommand{\it}[3]{\int_{#1}^{#2}#3\,dt}
% Usage '\ix{0}{1}{f(x)}'
\newcommand{\ix}[3]{\int_{#1}^{#2}#3\,dx}
% Usage '\iy{0}{1}{f(y)}'
\newcommand{\iy}[3]{\int_{#1}^{#2}#3\,dy}

% General integral with differential of choice:
% Usage '\id{0}{1}{f}{\mu}'
\newcommand{\id}[4]{\int_{#1}^{#2}#3\,d#4}
\newcommand{\idd}[4]{\int_{#1}^{#2}#3\,d^2#4}
\newcommand{\im}[4]{\int_{#1}^{#2}#3\,#4}

% Product: first parameter is bottom, second is top
% Usage: '\p{n=1}{\infty}'
\newcommand{\p}{\prime}

% Derivative: first parameter is function and second is variable
% Usage: '\der{f}{x}' or '\der{}{x}'
\newcommand{\der}[2]{\frac{d#1}{d#2}}

% Partial Derivative: first parameter is function and second is variable
% Usage: '\pder{f}{x}' or '\der{}{x}'
\newcommand{\pder}[2]{\frac{\partial #1}{\partial #2}}

% Big Union: first parameter is bottom and second is top
% Usage: '\bu{n=1}{\infty}'
\newcommand{\bu}[2]{\bigcup_{#1}^{#2}}

% Big Square Union: first parameter is bottom and second is top
% Usage: '\bsqu{n=1}{\infty}'
\newcommand{\bsqu}[2]{\bigsqcup_{#1}^{#2}}

% Big Intersection: first parameter is bottom and second is top
% Usage: '\bi{n=1}{\infty}'
\newcommand{\bi}[2]{\bigcap_{#1}^{#2}}

% Commands for big brackets
\renewcommand{\l}{\left}                                     
\renewcommand{\r}{\right}

% Double prime
\newcommand{\pp}{{\prime\prime}}

% Overline and underline
\newcommand{\ol}{\overline}                                  
\newcommand{\ul}{\underline}

% Inner products
\newcommand{\la}{\langle}                                    
\newcommand{\ra}{\rangle}                     
\newcommand{\inp}[2]{\la #1,#2\ra}                   % Usage: '\inp{f}{g}'

% Weak convergence
\newcommand{\wto}{\rightharpoonup}                   

% Lebesgue outer measure
\newcommand{\lom}{\mu^\ast}    

% i.o., e.v.
\newcommand{\io}{\text{ i.o.}}
\newcommand{\ev}{\text{ e.v.}}

% Subset and supset
\newcommand{\sbs}{\subset}                                   
\newcommand{\sps}{\supset}                    

% Measurable space and measure space
\newcommand{\sa}[2]{(#1,#2)}
\newcommand{\sam}[3]{(#1,#2,#3)}              

% QM commands
\newcommand{\bra}[1]{\la #1|}                        % Bra
\newcommand{\ket}[1]{|#1\ra}                         % Ket
\newcommand{\braket}[2]{\la #1|#2\ra}         % Braket

% Blackboard bold commands
\newcommand{\N}{\mathbb{N}}
\newcommand{\Z}{\mathbb{Z}}
\newcommand{\Q}{\mathbb{Q}}
\newcommand{\R}{\mathbb{R}}
\newcommand{\C}{\mathbb{C}}
\newcommand{\pr}{\mathbb{P}}
\newcommand{\E}{\mathbb{E}}
\newcommand{\Rs}{\mathbb{R}^\ast}
\newcommand{\T}{\mathbb{T}}
\newcommand{\RP}{\mathbb{RP}}
\newcommand{\D}{\mathbb{D}}
\renewcommand{\S}{\mathbb{S}}
\newcommand{\F}{\mathbb{F}}
\newcommand{\A}{\mathbb{A}}
\renewcommand{\P}{\mathbb{P}}

% For changing space around \[\], align*, and in-line math
\newcommand{\vm}{\vspace{-3mm}}
\newcommand{\ds}{\displaystyle}

% Greek letters
\newcommand{\g}{\gamma}
\renewcommand{\d}{\delta}
\renewcommand{\a}{\alpha}
\renewcommand{\b}{\beta}
\renewcommand{\epsilon}{\varepsilon}
\newcommand{\e}{\epsilon}
%\renewcommand{\phi}{\varphi}

\newcommand{\ti}{\widetilde}
\newcommand{\wh}{\widehat}

\renewcommand{\sp}{\text{sp}}

\newcommand{\Calg}{\textbf{C*-alg}}
\newcommand{\Ab}{\textbf{Ab}}

\renewcommand{\L}{\Lambda}

\newcommand{\eV}{\text{ eV}}
\newcommand{\MeV}{\text{ MeV}}
\newcommand{\meV}{\;\mu\text{eV}}
\newcommand{\J}{\text{ J}}

\renewcommand{\mod}[1]{\;\;(\text{mod }#1)}

\newcommand{\tl}{\triangleleft}

\newcommand{\Sc}{\text{sc}}
\newcommand{\Mp}{\text{mp}}

\newcommand{\me}[2]{\mu_{#1}^{#2}}

\newcommand{\indep}{\perp \!\!\! \perp}

% Math Operators: Not really sure why I just don't make these into commands, but it works out well so I'm not changing it.
\DeclareMathOperator*{\var}{Var}                     % Variance
\DeclareMathOperator*{\cov}{Cov}                     % Covariance
\DeclareMathOperator*{\corr}{Corr}                   % Correlation
\DeclareMathOperator*{\rank}{rank}                   % Rank
\DeclareMathOperator*{\nullity}{null}                % Nullity
\DeclareMathOperator*{\Id}{Id}                       % Identity
\DeclareMathOperator*{\Span}{span}                   % Span
\renewcommand{\Im}{\text{Im}}
\renewcommand{\Re}{\text{Re}}
\DeclareMathOperator*{\tr}{Tr}                       % Trace
\DeclareMathOperator*{\vol}{vol}                     % Volume
\DeclareMathOperator*{\cl}{cl}                       % Closure
\DeclareMathOperator*{\iso}{Iso}                     % Isometry Group
\DeclareMathOperator*{\interior}{Int}
\DeclareMathOperator*{\rad}{rad}
\DeclareMathOperator*{\Res}{Res}
\DeclareMathOperator*{\diag}{diag}
\DeclareMathOperator*{\GL}{GL}
\DeclareMathOperator*{\Hom}{Hom}
\DeclareMathOperator*{\Ad}{Ad }
\DeclareMathOperator*{\Tr}{Tr}
\DeclareMathOperator*{\Inv}{Inv}
\newcommand{\Invn}{{\Inv}_0}
\DeclareMathOperator*{\Trig}{Trig}
\DeclareMathOperator*{\Pol}{Pol}
\DeclareMathOperator*{\Lin}{Lin}
\DeclareMathOperator*{\Proj}{Proj}
\DeclareMathOperator*{\Char}{char}
\DeclareMathOperator*{\orb}{orb}
\DeclareMathOperator*{\End}{End}

% Mathematical Spaces
\newcommand{\Mat}{\text{Mat}}

% Random Distributions
\newcommand{\Exp}{\text{Exp}}
\newcommand{\Pois}{\text{Pois}}

% Theorem, lemma, etc. settings; Usage: '\begin{lemma} / \end{lemma}'
\theoremstyle{definition}
\newtheorem{theorem}{Theorem}[section]
\newtheorem{proposition}[theorem]{Proposition}
\newtheorem{lemma}[theorem]{Lemma}
\newtheorem{corollary}[theorem]{Corollary}
\newtheorem{definition}[theorem]{Definition}
\newtheorem{example}[theorem]{Example}
\newtheorem{remark}[theorem]{Remark}
\newtheorem{summary}[theorem]{Summary}
\newtheorem{properties}[theorem]{Properties}
\newtheorem{note}[theorem]{Note}

\title{KPZ Formula}
\author{Alexander Byard}
\date{\today}

% -- Document starts here --
\begin{document}

\maketitle

\section{Background for the KPZ Formula}

\begin{definition}[Gaussian Free Field]
The full plane GFF $\ol{X}$ is defined as a Gaussian random variable living in $\mc{S}^\p(\C)/\R$ with covariance
\[\E\l[\l(\idd{\C}{}{f(x)\ol{X}(x)}{x}\r)\l(\idd{\C}{}{h(x)\ol{X}(x)}{x}\r)\r]=\im{\C^2}{}{f(x)h(y)\ln\l(\frac{1}{|x-y|}\r)}{d^2x\,d^2y}\]
for all $f,h\in\mc{S}_0(\C)$.

The GFF $X$ with vanishing mean on the unit circle is given by
\[X(x)=\ol{X}(x)-\im{\C}{}{\ol{X}(x)}{\rho(d^2x)},\]
where $\rho$ is the uniform probability measure on the unit circle. The GFF $X$ has covariance
\[\E\l[\l(\idd{\C}{}{f(x)X(x)}{x}\r)\l(\idd{\C}{}{h(x)X(x)}{x}\r)\r]=\im{\C^2}{}{f(x)h(y)K(x,y)}{d^2x\,d^2y},\]
where
\[K(x,y)=\ln\l(\frac{1}{|x-y|}\r)+\ln(|x|_+)+\ln(|y|_+)\]
and $|x|_+:=|x|\vee1$.
\end{definition}

\begin{definition}[GMC Measure]
Let $\gamma\in(0,2)$. We denote by $X_\e$ the circle average approximation of the GFF $X$, i.e.
\[X_\e(x)=\frac{1}{2\pi}\id{0}{2\pi}{X(x+\e e^{i\theta})}{\theta}.\]
The GMC measure $M_\gamma(d^2x)$ is a random measure given by the following convergence in probability in the space of Radon measures:
\[M_\gamma(d^2x):=\lim_{\e\to0}e^{\gamma X_\e(x)-\frac{\gamma^2}{2}\E[X_\e(x)^2]}g(x)d^2x,\]
where $g$ is a conformal metric given by $g(x):=\frac{1}{|x|_+^4}$.
\end{definition}

\begin{definition}[Liouville Field]
We define the Liouville field $\phi$ to be
\[\phi=X+\frac{Q}{2}\ln(g)+c,\]
where $Q=\frac{\gamma}{2}+\frac{2}{\gamma}$. Moreover, we consider the measure $\E^Q$ given by
\[\E^Q[F(\phi)]:=\id{\R}{}{e^{-2Qc}\E\l[F\l(X+\frac{Q}{2}\ln(g)+c\r)\r]}{c},\]
where $F\in\mc{S}^\p(\C)$.
\end{definition}

The measure $\E^Q$ satisfies the following change of coordinate formula:

\begin{lemma}[Change of Coordinate Formula]\label{ChangeOfCoord}
For any M\"obius transform $\psi$, we have for all $F\in\mc{S}^\p(\C)$ that
\[\E^Q[F(\phi\circ\psi+Q\ln(|\psi^\p|))]=\E^Q[F(\phi)].\]
\end{lemma}

Next, let $\phi_\e$ be the circle average approximation of $\phi$, i.e.
\[\phi_\e(x)=\frac{1}{2\pi}\id{0}{2\pi}{\phi(x+\e e^{i\theta})}{\theta},\]
and consider the associated vertex operator
\[V_{\a,\e}(x):=\e^{\frac{\a^2}{2}}e^{\a\phi_\e(x)}.\]
We define the correlations of the vertex operators by
\[\l\la\prod_{k=1}^NV_{\a_i}(z_i)\r\ra_{\gamma,\mu}:=\lim_{\e\to0}\l\la\prod_{k=1}^NV_{\a_i,\e}(z_i)\r\ra_{\gamma,\mu},\]
where
\[\l\la\prod_{k=1}^NV_{\a_i,\e}(z_i)\r\ra_{\gamma,\mu}:=2\E^Q\l[e^{-\mu\idd{\C}{}{V_{\gamma,\e}(x)}{x}}\prod_{k=1}^NV_{\a_k,\e}(z_k)\r].\]

\begin{proposition}
The limit
\[\l\la\prod_{k=1}^NV_{\a_i}(z_i)\r\ra_{\gamma,\mu}:=\lim_{\e\to0}\l\la\prod_{k=1}^NV_{\a_i,\e}(z_i)\r\ra_{\gamma,\mu}\]
exists and equals
\[2\mu^{-s}\gamma^{-1}\Gamma(s)\prod_{1\leq i<j\leq N}\frac{1}{|z_i-z_j|^{\a_i\a_j}}\E\l[\l(\im{\C}{}{F(x,\bg{z})}{M_\gamma(d^2x)}\r)^{-s}\r],\]
where $s:=\frac{\s{k=1}{N}\a_k-2Q}{\gamma}$, and  $F(x,\bg{z}):=\prod_{k=1}^N\l(\frac{|x|_+}{|x-z_k|}\r)^{\gamma\a_k}$.
\end{proposition}

\section{KPZ Formula}

Given the facts above, we are ready to prove the KPZ formula.

\begin{proposition}[KPZ Formula]
For any M\"obius transform $\psi$, we have
\[\l\la\prod_{k=1}^NV_{\a_k}(\psi(z_k))\r\ra=\prod_{k=1}^N|\psi^\p(z_k)|^{-2\Delta_{\a_k}}\l\la\prod_{k=1}^NV_{\a_k}(z_k)\r\ra,\]
where $\Delta_\a=\frac{\a}{2}(Q-\frac{\a}{2})$ is the conformal weight.
\end{proposition}

\begin{proof}
Let $\e>0$ and let
\[F(\phi)=e^{-\mu\idd{\C}{}{\e^{\frac{\gamma^2}{2}}e^{\gamma\phi_\e(x)}}{x}}\prod_{k=1}^N\e^{\frac{\a_k^2}{2}}e^{\a_k\phi_\e(z_k)}.\]
By the change of coordinate formula found in Lemma \ref{ChangeOfCoord}, for any M\"obius transform $\psi$ we have
\begin{align*}
\E^Q[F(\phi)]
&=\E^Q\l[e^{-\mu\idd{\C}{}{\e^{\frac{\gamma^2}{2}}e^{\gamma\phi_\e(x)}}{x}}\prod_{k=1}^N\e^{\frac{\a_k^2}{2}}e^{\a_k\phi_\e(z_k)}\r]\\
&=\E^Q\l[e^{-\mu\e^{\frac{\gamma^2}{2}}\idd{\C}{}{e^{\gamma(\phi\circ\psi)_\e(x)+\gamma Q\ln(|\psi^\p(x)|)_\e}}{x}}\prod_{k=1}^N\e^{\frac{\a_k^2}{2}}e^{\a_k(\phi\circ\psi)_\e(z_k)+\a_kQ\ln(|\psi^\p(z_k)|)_\e}\r]\\
&=\E^Q[F(\phi\circ\psi+Q\ln(|\psi^\p|))].
\end{align*}
For small $\e$ we have $\ln(|\psi^\p(x)|)_\e\approx\ln(|\psi^\p(x)|)$. Moreover, notice that for any $\a$ we have
\[\a Q=2\frac{\a}{2}Q-\frac{\a^2}{2}+\frac{\a^2}{2}=2\frac{\a}{2}\l(Q-\frac{\a}{2}\r)+\frac{\a^2}{2}=2\Delta_\a+\frac{\a^2}{2}.\]
Hence, it follows that
\begin{align*}
\E^Q[F(\phi)]
&=\E^Q\l[e^{-\mu\idd{\C}{}{\e^{\frac{\gamma^2}{2}}e^{\gamma\phi_\e(x)}}{x}}\prod_{k=1}^N\e^{\frac{\a_k^2}{2}}e^{\a_k\phi_\e(z_k)}\r]\\
&\approx\E^Q\l[e^{-\mu\e^{\frac{\gamma^2}{2}}\idd{\C}{}{e^{\gamma(\phi\circ\psi)_\e(x)+\gamma Q\ln(|\psi^\p(x)|)}}{x}}\prod_{k=1}^N\e^{\frac{\a_k^2}{2}}e^{\a_k(\phi\circ\psi)_\e(z_k)+\a_kQ\ln(|\psi^\p(z_k)|)}\r]\\
&=\E^Q\l[e^{-\mu\e^{\frac{\gamma^2}{2}}\idd{\C}{}{e^{\gamma(\phi\circ\psi)_\e(x)+\ln(|\psi^\p(x)|^{\gamma Q})}}{x}}\prod_{k=1}^N\e^{\frac{\a_k^2}{2}}e^{\a_k(\phi\circ\psi)_\e(z_k)+\ln(|\psi^\p(z_k)|^{\a_kQ})}\r]\\
&=\E^Q\l[e^{-\mu\e^{\frac{\gamma^2}{2}}\idd{\C}{}{e^{\gamma(\phi\circ\psi)_\e(x)}|\psi^\p(x)|^{\gamma Q}}{x}}\prod_{k=1}^N\e^{\frac{\a_k^2}{2}}e^{\a_k(\phi\circ\psi)_\e(z_k)}|\psi^\p(z_k)|^{\a_kQ}\r]\\
&=\E^Q\l[e^{-\mu\e^{\frac{\gamma^2}{2}}\idd{\C}{}{e^{\gamma(\phi\circ\psi)_\e(x)}|\psi^\p(x)|^{2\Delta_\gamma+\frac{\gamma^2}{2}}}{x}}\prod_{k=1}^N\e^{\frac{\a_k^2}{2}}e^{\a_k(\phi\circ\psi)_\e(z_k)}|\psi^\p(z_k)|^{2\Delta_{\a_k}+\frac{\a_k^2}{2}}\r].
\end{align*}
Rearranging, we obtain the following expression for $\E^Q[F(\phi)]$:
\begingroup\makeatletter\def\f@size{9.5}\check@mathfonts
\[\l(\prod_{k=1}^N|\psi^\p(z_k)|^{2\Delta_{\a_k}}\r)\E^Q\l[e^{-\mu\idd{\C}{}{(|\psi^\p(x)|\e)^{\frac{\gamma^2}{2}}e^{\gamma(\phi\circ\psi)_\e(x)}|\psi^\p(x)|^{2\Delta_\gamma}}{x}}\prod_{k=1}^N(|\psi^\p(z_k)|\e)^{\frac{\a_k^2}{2}}e^{\a_k(\phi\circ\psi)_\e(z_k)}\r].\]
\endgroup
By definition of circle average we have
\[(\phi\circ\psi)_\e(x)\approx\phi_{|\psi^\p(x)|\e}(\psi(x)).\]
In particular, using $\Delta_\gamma=1$ we obtain
\begin{align*}
\idd{\C}{}{(|\psi^\p(x)|\e)^{\frac{\gamma^2}{2}}e^{\gamma(\phi\circ\psi)_\e(x)}|\psi^\p(x)|^{2\Delta_\gamma}}{x}
&=\idd{\C}{}{(|\psi^\p(x)|\e)^{\frac{\gamma^2}{2}}e^{\gamma(\phi\circ\psi)_\e(x)}|\psi^\p(x)|^2}{x}\\
&\approx\idd{\C}{}{(|\psi^\p(x)|\e)^{\frac{\gamma^2}{2}}e^{\gamma\phi_{|\psi^\p(x)|\e}(\psi(x))}|\psi^\p(x)|^2}{x}\\
&=\idd{\C}{}{(|\psi^\p(\psi^{-1}(u))|\e)^{\frac{\gamma^2}{2}}e^{\gamma\phi_{|\psi^\p(\psi^{-1}(u))|\e}(u)}}{u}\\
&\stackrel{\e\to0}{\to}M_\gamma(\C)
\end{align*}
by a change of variable $\e_u=|\psi^\p(\psi^{-1}(u))|\e$ as $\e\to0$. Using $(\phi\circ\psi)_\e(z_k)\approx\phi_{|\psi^\p(z_k)|_\e}(\psi(z_k))$ and a change of variable $\e_k=|\psi^\p(z_k)|\e$ at each vertex operator, we find
\begingroup\makeatletter\def\f@size{9.5}\check@mathfonts
\begin{align*}
\l(\prod_{k=1}^N|\psi^\p(z_k)|^{2\Delta_{\a_k}}\r)&\E^Q\l[e^{-\mu\idd{\C}{}{(|\psi^\p(x)|\e)^{\frac{\gamma^2}{2}}e^{\gamma(\phi\circ\psi)_\e(x)}|\psi^\p(x)|^{2\Delta_\gamma}}{x}}\prod_{k=1}^N(|\psi^\p(z_k)|\e)^{\frac{\a_k^2}{2}}e^{\a_k(\phi\circ\psi)_\e(z_k)}\r]\\
&\stackrel{\e\to0}{\to}\frac{1}{2}\l(\prod_{k=1}^N|\psi^\p(z_k)|^{2\Delta\a_k}\r)\l\la\prod_{k=1}^NV_{\a_k}(\psi(z_k))\r\ra.
\end{align*}
\endgroup
\end{proof}




































































\end{document}
